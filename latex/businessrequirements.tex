\section{Business Requirements}\label{sec:businessrequirements}

The operating model of the SmartV platform relies heavily on ready access to large amounts of storage, compute, and networking capability. This is especially true given the large amounts of high resolution video data the platform is expected to process and store. The business requirements captured from this understanding are presented here in two sections: the non-functional requirements relating to SmartV's operations and the functional requirements relating to the cloud infrastructure specifically.

\subsection{Operational Requirements}

The requirements outlined here represent non-functional and operational functions of the cloud infrastructure solution.

\subsubsection*{Scalability}

Scalability represents a primary business objective for SmartV's transition to cloud-based infrastructure, and underlies most of the business requirements outlined in this document. The deployment of scalable infrastructure provides SmartV the ability to react to fluctuations in platform usage while minimising any costs associated with under-utilised infrastructure.

\subsubsection*{Availability}

The SmartV platform's customer-facing model requires services to be highly available to end-users, ensuring uninterrupted access to videos and features. Services need to be available even during peak usage times or unforeseen spikes in demand.

% S3 has 4 9's of durability https://docs.aws.amazon.com/AmazonS3/latest/userguide/DataDurability.html

\subsubsection*{Reliability}

The cloud infrastructure used to enable SmartV must also prioritise high availability, with resilient architecture that minimises downtime and ensures continuous service delivery. 

\subsubsection*{Security}

A high level of security is required to protect user data and the SmartV platform itself. All data hosted by the cloud infrastructure should be encrypted and subject to strict access control and authentication. Both ingress and egress traffic should be restricted with additional monitoring across the network and infrastructure enabling incident detection and response. 

\subsection{Infrastructure Requirements}

The functional requirements of the cloud infrastructure itself form the core project, with each item listed here built on the operational requirements outlined above.

\subsubsection*{Storage}

Effective storage management is a key requirement of SmartV's infrastructure, with the service housing massive amounts of multi-media data for its users. All stored data also needs to be safely and regularly replicated for back-ups and disaster recovery.


\subsubsection*{Compute}

The computational intensity of video based workloads necessitates significant amounts of domain-specific compute availability in the form of graphics processing unit (GPU) access, as well as the additional components required to support them. 

\subsubsection*{Network}

High throughput and reliable networking is key to effective delivery of SmartV services.

\subsubsection*{Infrastructure Management}

Deployed cloud infrastructure should be fully managed and configurable in anticipation of changes in future requirements, deployment of new services, and regular updates and maintenance.


\subsection{Cloud Provider Selection}
The growth in popularity of cloud services has fostered an industry with many competing cloud providers. Given the continued growth of the SmartV user-base, the choice to leverage an existing provider has been made to balance the increasing but dynamic demands of the platform while maintaining minimal-cost cloud infrastructure. Both internal and public cloud providers have been considered, including offerings from VMware, OpenStack, Azure, and Amazon Web Services (AWS).

The use of internal cloud provisioning necessitates a significant up-front cost in both the hardware and software required by the transition to cloud infrastructure, and is over-all less flexible than the infrastructure-as-a-service models provided by Azure and AWS. These caveats to an internal cloud deployment represent unnecessary additional costs and management responsibilities to SmartV, reducing the organisations focus on end-users.

For these reasons and over-all compatibility with the business requirements outlined in section \ref{sec:businessrequirements} AWS is selected as the cloud provider of choice for the project. The use of AWS also allows SmartV to bypass the upfront cost of any immediate upgrades to physical infrastructure by instead utilising storage, compute, and network provided by the AWS platform, paying for only the resources used by SmartV. The choice of AWS is also influenced by the ready availability of high-power compute instances required by the SmartV platform (see section \ref{sec:infrastructure-components}).
