
\section{Cloud Computing}

Cloud computing technology allows users to access and use computing resources (servers, storage, software, databases, computational power, etc) over the internet, or the ‘cloud’ as it is commonly called. This removes the overhead of needing to own and maintain physical hardware and infrastructure on-site, and can allow businesses to grow at their own pace.

There are 5 primary components of cloud computing: servers, networks, virtualisation software (Virtual Machines), hypervisors, and storage.

\begin{itemize}
    \item \textbf{Servers} \\ The servers of the cloud are places that the users can connect to in order to complete their work. These have the computational power in order to host many virtual machines and different applications

    \item \textbf{Networks} \\ Network infrastructure allows users to connect to other components of the infrastructure. This is essential for cloud computing as this allows for the virtualisation of the different components of the cloud, and includes routers, switches, load balancers, and cables.

    \item \textbf{Virtualisation Software (Virtual Machines)} \\ Stand-alone devices that use the resources on servers to create a new device. These allow users to run applications and do work on devices without being directly tied to the hardware that they are using.

    \item \textbf{Hypervisors} \\ A hypervisor is the manager of any operating systems of the guests and also manages any system resources that the guests will use, such as CPUs, memory, and storage. This allows many virtual machines to run on a single physical machine. Additionally, two different types of hypervisor allow for a more tailored experience to what you require. A Type 1 hypervisor will directly access the resources of the machine it is on. Comparatively, Type 2 share and negotiate resourse allocation with the underlying operating system of the machine.

    \item \textbf{Storage} \\ Storage allows users to store their data in a safe place, and can include object storage (files, images, videos), or block storage (VMs, databases). As well as this, storage virtualisation can be used in order to combine the storage of multiple places/devices into a single storage pool, making it easier to manage significant amounts of data.

\end{itemize}

